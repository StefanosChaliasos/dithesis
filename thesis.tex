% demo.tex
%
% Enjoy, evolve, and share!
%
% Compile it as follows:
%   latexmk
%
% Check file `dithesis.cls' for other configuration options.
%
\documentclass[inscr,ack,preface]{dithesis}

%\usepackage{graphicx}

%%%%%%%%%%%%%%%%%%%%%%%%%%%%%%%%%%%%%%%%%%%%%%%%%%%%%%%%%%%%%%%%%%%%%%%%%%%%%%%
%%%%%%%%%%%%%%%%%%%% User-specific package inclusions %%%%%%%%%%%%%%%%%%%%%%%%%
%%%%%%%%%%%%%%%%%%%%%%%%%%%%%%%%%%%%%%%%%%%%%%%%%%%%%%%%%%%%%%%%%%%%%%%%%%%%%%%
\usepackage{booktabs}
\usepackage{hyperref}
\usepackage{lipsum}
\usepackage{enumerate}
\usepackage{amsmath}
\usepackage{amssymb}
\hypersetup{
    unicode=true,                     % non-Latin characters in bookmarks
    pdffitwindow=true,                % page fit to window when opened
    pdfnewwindow=true,                % links in new window
    pdfkeywords={},                   % list of keywords
    colorlinks=true,                  % false: boxed links; true: colored links
    linkcolor=black,                  % color of internal links
    citecolor=black,                  % color of links to bibliography
    filecolor=black,                  % color of file links
    urlcolor=black,                   % color of external links
    pdftitle={},                      % title
    pdfauthor={},                     % author
    pdfsubject={}                     % subject of the document
}
%%%%%%%%%%%%%%%%%%%%%%%%%%%%%%%%%%%%%%%%%%%%%%%%%%%%%%%%%%%%%%%%%%%%%%%%%%%%%%%
%%%%%%%%%%%%%%%%%%%% User-specific package inclusions %%%%%%%%%%%%%%%%%%%%%%%%%
%%%%%%%%%%%%%%%%%%%%%%%%%%%%%%%%%%%%%%%%%%%%%%%%%%%%%%%%%%%%%%%%%%%%%%%%%%%%%%%


%%%%%%%%%%%%%%%%%%%%%%%%%%%%%%%%%%%%%%%%%%%%%%%%%%%%%%%%%%%%%%%%%%%%%%%%%%%%%%%
%%%%%%%%%%%%%%%%%%%%%% User-specific configuration %%%%%%%%%%%%%%%%%%%%%%%%%%%%
%%%%%%%%%%%%%%%%%%%%%%%%%%%%%%%%%%%%%%%%%%%%%%%%%%%%%%%%%%%%%%%%%%%%%%%%%%%%%%%
%%%%%%%%%%%%%%%%%%%%%%%%%%%%%%%%%%%%%%%%%%%%%%%%%%%%%%%%%%%%%%%%%%%%%%%%%%%%%%%
%%%%%%%%%%%%%%%%%%%%%% User-specific configuration %%%%%%%%%%%%%%%%%%%%%%%%%%%%
%%%%%%%%%%%%%%%%%%%%%%%%%%%%%%%%%%%%%%%%%%%%%%%%%%%%%%%%%%%%%%%%%%%%%%%%%%%%%%%


%%%%%%%%%%%%%%%%%%%%%%%%%%%%%%%%%%%%%%%%%%%%%%%%%%%%%%%%%%%%%%%%%%%%%%%%%%%%%%%
%%%%%%%%%%%%%%%%%%%%%%%%%%% Required Metadata %%%%%%%%%%%%%%%%%%%%%%%%%%%%%%%%%
%%%%%%%%%%%%%%%%%%%%%%%%%%%%%%%%%%%%%%%%%%%%%%%%%%%%%%%%%%%%%%%%%%%%%%%%%%%%%%%
%
% First name, last name
%
\authorFirstGr{Όνομα}
\authorFirstAbrGr{Ο.} % abbreviation of first name
\authorMiddleGr{Σ.}   % abbreviation of father's first name
\authorLastGr{Επίθετο}
\authorFirstEn{Onoma}
\authorFirstAbrEn{O.}
\authorMiddleEn{S.}
\authorLastEn{Epitheto}
\authorSn{1115000000000}

%
% The title of the thesis
%
\titleEn{BSc thesis title in English}
\titleGr{Τίτλος της πτυχιακής εργασίας στα Ελληνικά}

%
% Month followed by Year
%
\dateGr{ΜΑΪΟΣ 2010}
\dateEn{MAY 2010}

%
% Supervisor(s) info
%
\supervisorGr{Όνομα Επώνυμο}{Βαθμίδα}
\supervisorEn{Name Surname}{Rank}

%
% Abstract, synopsis, inscription, ack, and preface pages.
%
\abstractEn{
  \begin{align*}
    G \leftarrow A_1 \land \dots \land A_n \\
    \mathsf{G \leftarrow A_1 \land \dots \land A_n}
  \end{align*}
\begin{greek}
Η περίληψη, η επιστημονική περιοχή και οι λέξεις κλειδιά στα Αγγλικά. Δεν θα πρέπει να υπερβαίνει τη μία σελίδα.
\end{greek}
}
\abstractGr{
\begin{greek}
Η περίληψη περιλαμβάνει το σκοπό-αντικείµενο της εργασίας, τη μεθοδολογία, τα κύρια βήματα που ακολουθήθηκαν και τέλος τα κύρια αποτελέσματα. Μετά το τέλος της περίληψης θα δηλώνεται η επιστημονική περιοχή της εργασίας και 5 λέξεις κλειδιά. Η συνολική έκταση της περίληψης και των λέξεων δήλωσης επιστημονικής περιοχής και λέξεων-κλειδιών θα είναι μέχρι µία σελίδα. Δεν θα πρέπει να υπερβαίνει τη μία σελίδα.
\end{greek}
}
\acksEn{
\begin{greek}
Στη σελίδα αυτή αναφέρονται οι ευχαριστίες. Η σελίδα αυτή είναι προαιρετική. Παρατίθεται παράδειγμα ευχαριστιών.

Για τη διεκπεραίωση της παρούσας Πτυχιακής Εργασίας, θα θέλαμε να ευχαριστήσουμε τους επιβλέποντες, αν. καθ .Ευστράτιο Γεωργιάδη, Γρηγόριο Παπάμαλο, Αναστασία Γούσιου, Ξενοφών Παπαδόπουλο, για τη συνεργασία και την πολύτιμη συμβολή του στην ολοκλήρωση της.
\end{greek}
}
\prefaceEn{
\begin{greek}
Στον πρόλογο αναφέρονται θέματα που δεν είναι επιστημονικά ή τεχνικά, όπως το πλαίσιο που διενεργήθηκε η εργασία, ευχαριστίες, ο τόπος διεξαγωγής κλπ.
\end{greek}
}

\inscriptionEn{\emph{Στη σελίδα αυτή αναφέρονται οι αφιερώσεις. Η σελίδα αυτή είναι προαιρετική.}}

%
% Subject area and keywords
%
\subjectAreaGr{Θεματική Περιοχή}
\subjectAreaEn{Subject Area}
\keywordsGr{Λέξη κλειδί1, λέξη κλειδί2, λέξη κλειδί3}
\keywordsEn{Keyword1, keyword2, keyword3}

%
% Set the .bib file containing your paper publications (leave the extension out)
%
% This is optional, but it should be specified when option 'lop' is passed to
% the document class.
%
% Then, inside the document environment, you may use the command '\nocitelop' to
% site your papers, as you would traditionally do with the commands '\cite' or
% '\nocite'.
%
% The papers are printed in reverse chronological order.
%
%\lopfile{mypapers/pubs}
%%%%%%%%%%%%%%%%%%%%%%%%%%%%%%%%%%%%%%%%%%%%%%%%%%%%%%%%%%%%%%%%%%%%%%%%%%%%%%%
%%%%%%%%%%%%%%%%%%%%%%%%%%% Required Metadata %%%%%%%%%%%%%%%%%%%%%%%%%%%%%%%%%
%%%%%%%%%%%%%%%%%%%%%%%%%%%%%%%%%%%%%%%%%%%%%%%%%%%%%%%%%%%%%%%%%%%%%%%%%%%%%%%

\begin{document}

\frontmatter

\mainmatter

% add main chapters (should be given in capital letters)
\chapter{INTRODUCTION}
  \begin{greek}
    \section{Μορφοποίηση Κειμένου}
    Το παρόν αρχείο αποτελεί το υπόδειγμα (template) για τη μορφοποίηση της εργασίας.

    Για την ομοιόμορφη εμφάνιση των σχετικών τόμων και του ψηφιακού υλικού που θα παραδίδονται στην Βιβλιοθήκη, το Τμήμα καθιερώνει υποχρεωτικά πρότυπα. Για το σκοπό αυτό θα πρέπει να τηρούνται αυστηρώς οι οδηγίες που παρατίθενται στη συνέχεια.

    \subsection{Μέγεθος σελίδας}
    Το μέγεθος της σελίδας θα πρέπει να είναι \textbf{A4}.

    \subsection{Ημερομηνίες}
    Τα στοιχεία του μηνός και του έτους που θα αναγράφονται στη εργασία είναι αυτά της ημερομηνίας εξέτασης. Τα ίδια ημερομηνιακά στοιχεία θα αναγράφονται και σε οποιοδήποτε συνοδευτικό υλικό κατατίθεται στη Βιβλιοθήκη.

    \subsection{Εξώφυλλο και \grnumm{1} Εσώφυλλο (Σελίδα Τίτλου)}
    Όπως στην αρχή του παρόντος προτύπου. Δηλαδή με τη σειρά:
    \begin{enumerate}
      \item Εικονίδιο της Αθηνάς: άνω στο κέντρο.
      \item Τίτλος του Πανεπιστημίου: Arial έντονα κεφαλαία 14.
      \item Τίτλος Σχολής Arial έντονα κεφαλαία 12.
      \item Τίτλος του Τμήματος: Arial έντονα κεφαλαία 12.
      \item Είδος εργασίας (Πτυχιακή Εργασία) Arial έντονα κεφαλαία 12.
      \item Τίτλος της Εργασίας: Arial έντονα πεζά 16.
      \item Όνομα, αρχικό γράμμα πατρώνυμου και επώνυμο φοιτητή: Arial έντονα πεζά 12.
      \item \underline{Μόνο στην περίπτωση των \textbf{Πτυχιακών} και \textbf{Διπλωματικών} εργασιών}

            Επιβλέπων (ή Επιβλέπουσα) ή Επιβλέποντες (ή Επιβλέπουσες): Όνομα και επώνυμο καθηγητή Arial πεζά έντονα 12, τίτλος καθηγητή Arial πεζά 12. Στην περίπτωση που υπάρχουν συνεπιβλέποντες μη μέλη ΔΕΠ προστίθενται στην αμέσως επόμενη γραμμή κατά τον ίδιο τρόπο με τους επιβλέποντες καθηγητές.
      \item Τόπος ολοκλήρωσης της εργασίας (που είναι πάντα ΑΘΗΝΑ): Arial έντονα κεφαλαία 12.
      \item Μήνας και έτος ολοκλήρωσης της εργασίας: Arial έντονα κεφαλαία 12. Θα είναι ο μήνας και το έτος εξέτασης της εργασίας.
      \item Το διάστιχο στα στοιχεία του εξωφύλλου και 1ου εσώφυλλο θα πρέπει να είναι 1pt.
      \item Η αρίθμηση των σελίδων αρχίζει νοητά από το 1ο εσώφυλλο (σελίδα τίτλου), χωρίς όμως να αναγράφεται ο αριθμός της σελίδας σε αυτό. Η αρίθμηση των σελίδων θα αρχίσει να φαίνεται από την 1η σελίδα του πίνακα περιεχομένων και μετά.
      \item Το πίσω μέρος της σελίδας αυτής παραμένει λευκό.
    \end{enumerate}

    \subsection{\grnumm{2} Εσώφυλλο (Σελίδα έγκρισης)}
    Όπως στην αρχή του παρόντος προτύπου. Δηλαδή με τη σειρά:

    Είδος εργασίας: \textbf{ΠΤΥΧΙΑΚΗ ΕΡΓΑΣΙΑ} Arial έντονα κεφαλαία 12

    Τίτλος: Arial πεζά 12

    Κέντρο:
    \begin{itemize}
      \item Όνομα και επώνυμο φοιτητή: Arial έντονα πεζά 12
      \item Αριθμός Μητρώου (Α. Μ.) του φοιτητή (μόνο για τις πτυχιακές και τις μεταπτυχιακές εργασίες): Arial κεφαλαία 1
    \end{itemize}

    Αριστερά:

    «Επιβλέπων (ή Επιβλέπουσα ή Επιβλέποντες ή Επιβλέπουσες)» (για τις πτυχιακές εργασίες).
    \begin{itemize}
      \item Arial έντονα κεφαλαία 12
      \item Τίτλος Καθηγητή: Arial πεζά 12
      \item Όνομα και Επώνυμο Καθηγητή: Arial έντονα πεζά 12
    \end{itemize}

    Το πίσω μέρος της σελίδας αυτής παραμένει λευκό.

    \subsection{Περίληψη}
    Μετά το \grnumm{2} εσώφυλλο θα ακολουθούν σε δύο χωριστά φύλλα η περίληψη της εργασίας στην ελληνική γλώσσα και η περίληψη της εργασίας στην αγγλική. Η περίληψη περιλαμβάνει το σκοπό-αντικείμενο της εργασίας, τη μεθοδολογία, τα κύρια βήματα που ακολουθήθηκαν και τέλος τα κύρια αποτελέσματα.

    Μετά το τέλος της περίληψης θα δηλώνεται η θεματική περιοχή της εργασίας και 5 λέξεις κλειδιά (ελληνικά και αγγλικά αντίστοιχα για κάθε σελίδα). Η συνολική έκταση της περίληψης και των λέξεων δήλωσης επιστημονικής περιοχής και λέξεων κλειδιών θα είναι μέχρι μία σελίδα (δείτε και σελίδες 3 και 4 στο παρόν υπόδειγμα).

    Το πίσω μέρος των σελίδων αυτών παραμένει λευκό.

    Στην περίπτωση της \underline{πτυχιακής} εργασίας, η περίληψη, η επιστημονική περιοχή και οι λέξεις κλειδιά στην αγγλική γλώσσα είναι προαιρετικά.

    Ακολουθούν σε χωριστές σελίδες, όπως και στο παρόν υπόδειγμα:

    \textbf{Αφιερώσεις} \textit{(προαιρετικά - το πίσω μέρος της σελίδας αυτής παραμένει λευκό)}

    \textbf{Ευχαριστίες} \textit{(προαιρετικά - το πίσω μέρος της σελίδας αυτής παραμένει  λευκό)}

    \textbf{Περιεχόμενα}

    \textbf{Πρόλογος} (Όπου αναφέρονται θέματα που δεν είναι επιστημονικά ή τεχνικά, όπως το πλαίσιο που διενεργήθηκε η εργασία, ευχαριστίες, ο τόπος διεξαγωγής κλπ.)

    \subsection{Αρίθμηση σελίδων}
    Η αρίθμηση των σελίδων πάντοτε αρχίζει νοητά από το \grnumm{1} εσώφυλλο (σελίδα τίτλου, η \grnumf{1} σελίδα του παρόντος υποδείγματος) χωρίς δηλαδή να αναγράφεται ο αριθμός της σελίδας σε αυτό. Και στο \grnumm{2} εσώφυλλο (σελίδα έγκρισης) επίσης ο αριθμός της σελίδας υπολογίζεται χωρίς  να αναγράφεται σε αυτό. Η αρίθμηση πάντοτε τελειώνει στην τελευταία τυπωμένη σελίδα. Θα πρέπει να αρχίσει να αναγράφεται ο αριθμός σελίδων από την πρώτη σελίδα του πρώτου κεφαλαίου (όπως στο παρόν υπόδειγμα).

    \subsection{Οι σελίδες του Κειμένου}
    Όπως στο παρόν υπόδειγμα. Δηλαδή:
    \begin{itemize}
      \item \textbf{Περιθώρια (Margins):}
      \begin{enumerate}[$\circ$]
        \item Άνω (Top): 2 cm
        \item Κάτω (Bottom): 2 cm
        \item Περιθώριο Βιβλιοδεσίας (Gutter): 0.5 cm
        \item Αριστερά (Left): 2 cm
        \item Δεξιά (Right): 2 cm
      \end{enumerate}
      \item \textbf{Κεφαλίδα (Header):} 1.25 cm (από πάνω): Ο τίτλος της εργασίας \emph{(δεν εισάγεται κεφαλίδα στο εξώφυλλο, στο 1ο και 2ο εσώφυλλο, στις σελίδες των περιλήψεων, στις σελίδες των αφιερώσεων και των ευχαριστιών και στις τυχόν λευκές σελίδες)}.
      \item \textbf{Υποσέλιδο (Footer):} 1.25 cm (από κάτω): Το όνομα ή τα ονόματα των συγγραφέων και ο αριθμός σελίδας \emph{(δεν εισάγεται υποσέλιδο στο εξώφυλλο, στο 1ο και 2ο εσώφυλλο, στις σελίδες των περιλήψεων, στις σελίδες των αφιερώσεων και των ευχαριστιών και στις τυχόν λευκές σελίδες)}.
      \item \textbf{Αρίθμηση σελίδας:} Δεξιά του υποσέλιδου και στην περίπτωση εκτύπωσης και από τις δύο πλευρές του φύλλου στο κέντρο του υποσέλιδου (Προσοχή: στο παρόν υπόδειγμα η αρίθμηση έχει γίνει στα δεξιά του υποσέλιδου, για εκτύπωση στη μία σελίδα του φύλλου). Το παρόν πρόπυπο υποστηρίζει την επιλογή \texttt{dualpage}, η οποία ρυθμίζει το υποσέλιδο για εκτύπωση και στις δύο πλευρές του φύλλου. Το μέγεθος γραμματοσειράς για την αρίθμηση της σελίδας θα πρέπει να είναι 10.
      \item \textbf{Μορφή Παραγράφου (Format Paragraph)}
      \begin{enumerate}[$\circ$]
        \item \textbf{Στοίχιση (Justification):} αριστερά και δεξιά
        \item \textbf{Διάκενο μεταξύ παραγράφων (paragraph spacing):}

        πριν: 0 στιγμές, μετά: 6 στιγμές
        \item \textbf{Διάστιχο (Line spacing):} 1 γραμμή
      \end{enumerate}
      \item \textbf{Γραμματοσειρά (Font):} Arial 12.
      \item \textbf{Τύπος Γραμματοσειράς (Font style):} Normal ή Regular.
      \item \textbf{Αρίθμηση Κεφαλαίων:} Arial 12 ή 14.
      \item \textbf{Τύπος Αρίθμησης Κεφαλαίων:} όπως στο παρόν υπόδειγμα.
      \item \textbf{Τίτλος Κεφαλαίων:} Κεφαλαία έντονα Arial 14, στοίχιση στο κέντρο.
      \item \textbf{Τίτλος Υποκεφαλαίων:} Έντονα (Bold) πεζά Arial 12, στοίχιση αριστερά.
      \item \textbf{Σχήματα/Διαγράμματα:} Κάθε σχήμα/διαγράμμα θα πρέπει να έχει υποχρεωτικά μοναδική αρίθμηση, είτε στο σύνολο της εργασίας είτε ανά κεφάλαιο, και οπωσδήποτε λεζάντα στο κάτω μέρος τους (τα διαγράμματα ανήκουν στην κατηγορία των σχημάτων), με στοίχιση όπως στο παρόν υπόδειγμα.
      \item \textbf{Εικόνες/Φωτογραφίες:} Όλες οι εικόνες/φωτογραφίες θα πρέπει να έχουν υποχρεωτικά μοναδική αρίθμηση και οπωσδήποτε λεζάντα στο κάτω μέρος τους, όπως στο παρόν υπόδειγμα.
      \item \textbf{Πίνακες:} Όλοι οι πίνακες πρέπει να φέρουν μοναδική αρίθμηση και λεζάντα στο πάνω μέρος τους, όπως στο παρόν υπόδειγμα.
    \end{itemize}

    \subsection{Ορολογία}
    Σε περίπτωση συγγραφής στα Ελληνικά, την πρώτη φορά που θα εμφανίζεται στο κείμενο ένας επιστημονικός όρος ο οποίος προέρχεται από μεταφρασμένο ξένο όρο θα αναφέρεται δίπλα σε παρένθεση ο αντίστοιχος ξενόγλωσσος όρος. Στο τέλος του κειμένου θα υπάρχει πίνακας ορολογίας με τις αντιστοιχίσεις των ελληνικών και ξενόγλωσσων όρων. Ως παράδειγμα παράθεσης ορολογίας δίνεται η εξής πρόταση: Ήδη από το 1994 η BELL ξεκίνησε στα εργαστήρια της προσπάθειες για τη σχεδίαση υπολογιστών µε αυξημένη αξιοπιστία (reliability). Δείτε και τον Πίνακα Ορολογίας στο παρόν υπόδειγμα.

    \subsection{Συντμήσεις - Αρκτικόλεξα}
    Στο τέλος του κειμένου θα υπάρχει «Πίνακας Συντμήσεων – Αρκτικόλεξων» όπου θα αναφέρονται οι συντμήσεις-αρκτικόλεξα και δίπλα ή πλήρη ανάπτυξη των ονομασιών. Αν, για παράδειγμα χρησιμοποιήσετε τον όρο W3C στο κείμενό σας, θα πρέπει να παραθέσετε την πλήρη ανάπτυξή του όπως στον Πίνακα Συντμήσεων – Αρκτικόλεξων στο παρόν υπόδειγμα.

    \subsection{Βιβλιογραφικές Αναφορές}
    Οι βιβλιογραφικές αναφορές έχουν ρυθμιστεί στο παρόν πρότυπο, έτσι ώστε να συμφωνούν με τις υποδείξεις του IEEE.

    Μέσα στο κείμενο, οι αναφορές γίνονται γράφοντας \verb!\cite{X}!, όπου \texttt{X} είναι το αναγνωριστικό της πηγής.

    \section{Άλλες Παρατηρήσεις}
    Θα πρέπει να ακολουθείτε το παρόν υπόδειγμα, όσον αφορά τη μορφοποίηση (εξώφυλλα, εσώφυλλα, κλπ) της εργασίας, τις κενές σελίδες, τα περιθώρια της σελίδας, της κεφαλίδας και του υποσέλιδου, τη μορφή της παραγράφου και των γραμματοσειρών, τις λεζάντες σε σχήματα, εικόνες και πίνακες, τη μοναδική αρίθμηση των λεζάντων και ό,τι άλλο εμφανίζεται στο παρόν υπόδειγμα. Επιπλέον, ιδιαίτερη προσοχή δώστε και στις παρακάτω παρατηρήσεις.

    \subsection{Λεζάντες}
    Κάθε σχήμα, διάγραμμα, εικόνα, φωτογραφία και πίνακας θα πρέπει να έχει υποχρεωτικά μοναδική αρίθμηση, είτε στο σύνολο της εργασίας είτε ανά κεφάλαιο, και οπωσδήποτε λεζάντα, όπως φαίνεται πιο πάνω, στο παρόν υπόδειγμα. \textbf{Προσοχή:} για τους πίνακες, η λεζάντα θα πρέπει να βρίσκεται επάνω από τον πίνακα.

    \subsection{Κεφαλίδες και Υποσέλιδα}
    \textbf{Δεν εισάγονται} στο εξώφυλλο, στο \grnumm{1} και \grnumm{2} εσώφυλλο, στις σελίδες των περιλήψεων, στις σελίδες των αφιερώσεων και των ευχαριστιών και στις τυχόν λευκές σελίδες. Εισάγονται από την \grnumf{1} σελίδα του 1ου κεφαλαίου και μετά

  \end{greek}


\chapter{BACKGROUND AND RELATED WORK}
  \lipsum[4-7]

\chapter{ANOTHER CHAPTER}
  \lipsum[10-20]

\chapter{CONCLUSIONS AND FUTURE WORK}
  \lipsum[3-5]

\backmatter

% abbreviations table
\abbreviations
\begin{center}
	\renewcommand{\arraystretch}{1.5}
	\begin{longtable}{ l @{\qquad} l }
	\toprule
	RDF    & Resource Description Framework \\
	SPARQL & SPARQL Protocol and RDF Query Language \\
	OWL    & Web Ontology Language \\
	OGC    & Open Geospatial Consortium \\
	\bottomrule
	\end{longtable}
\end{center}

% appendix
\begin{appendix}
% mark the beginning of the appendix
\appendixstartedtrue

% add appendix line to ToC
\phantomsection
\addcontentsline{toc}{chapter}{APPENDICES}

\chapter{FIRST APPENDIX}
\chapter{SECOND APPENDIX}
\chapter{THIRD APPENDIX}
\end{appendix}

% manually include the bibliography
\bibliographystyle{plain}
\bibliography{references}
% include it also in ToC (do sth on your own)
\addcontentsline{toc}{chapter}{REFERENCES}

\end{document}
